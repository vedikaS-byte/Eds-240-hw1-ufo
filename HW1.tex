% Options for packages loaded elsewhere
\PassOptionsToPackage{unicode}{hyperref}
\PassOptionsToPackage{hyphens}{url}
\PassOptionsToPackage{dvipsnames,svgnames,x11names}{xcolor}
%
\documentclass[
  letterpaper,
  DIV=11,
  numbers=noendperiod]{scrartcl}

\usepackage{amsmath,amssymb}
\usepackage{lmodern}
\usepackage{iftex}
\ifPDFTeX
  \usepackage[T1]{fontenc}
  \usepackage[utf8]{inputenc}
  \usepackage{textcomp} % provide euro and other symbols
\else % if luatex or xetex
  \usepackage{unicode-math}
  \defaultfontfeatures{Scale=MatchLowercase}
  \defaultfontfeatures[\rmfamily]{Ligatures=TeX,Scale=1}
\fi
% Use upquote if available, for straight quotes in verbatim environments
\IfFileExists{upquote.sty}{\usepackage{upquote}}{}
\IfFileExists{microtype.sty}{% use microtype if available
  \usepackage[]{microtype}
  \UseMicrotypeSet[protrusion]{basicmath} % disable protrusion for tt fonts
}{}
\makeatletter
\@ifundefined{KOMAClassName}{% if non-KOMA class
  \IfFileExists{parskip.sty}{%
    \usepackage{parskip}
  }{% else
    \setlength{\parindent}{0pt}
    \setlength{\parskip}{6pt plus 2pt minus 1pt}}
}{% if KOMA class
  \KOMAoptions{parskip=half}}
\makeatother
\usepackage{xcolor}
\usepackage[normalem]{ulem}
\setlength{\emergencystretch}{3em} % prevent overfull lines
\setcounter{secnumdepth}{-\maxdimen} % remove section numbering
% Make \paragraph and \subparagraph free-standing
\ifx\paragraph\undefined\else
  \let\oldparagraph\paragraph
  \renewcommand{\paragraph}[1]{\oldparagraph{#1}\mbox{}}
\fi
\ifx\subparagraph\undefined\else
  \let\oldsubparagraph\subparagraph
  \renewcommand{\subparagraph}[1]{\oldsubparagraph{#1}\mbox{}}
\fi

\usepackage{color}
\usepackage{fancyvrb}
\newcommand{\VerbBar}{|}
\newcommand{\VERB}{\Verb[commandchars=\\\{\}]}
\DefineVerbatimEnvironment{Highlighting}{Verbatim}{commandchars=\\\{\}}
% Add ',fontsize=\small' for more characters per line
\usepackage{framed}
\definecolor{shadecolor}{RGB}{241,243,245}
\newenvironment{Shaded}{\begin{snugshade}}{\end{snugshade}}
\newcommand{\AlertTok}[1]{\textcolor[rgb]{0.68,0.00,0.00}{#1}}
\newcommand{\AnnotationTok}[1]{\textcolor[rgb]{0.37,0.37,0.37}{#1}}
\newcommand{\AttributeTok}[1]{\textcolor[rgb]{0.40,0.45,0.13}{#1}}
\newcommand{\BaseNTok}[1]{\textcolor[rgb]{0.68,0.00,0.00}{#1}}
\newcommand{\BuiltInTok}[1]{\textcolor[rgb]{0.00,0.23,0.31}{#1}}
\newcommand{\CharTok}[1]{\textcolor[rgb]{0.13,0.47,0.30}{#1}}
\newcommand{\CommentTok}[1]{\textcolor[rgb]{0.37,0.37,0.37}{#1}}
\newcommand{\CommentVarTok}[1]{\textcolor[rgb]{0.37,0.37,0.37}{\textit{#1}}}
\newcommand{\ConstantTok}[1]{\textcolor[rgb]{0.56,0.35,0.01}{#1}}
\newcommand{\ControlFlowTok}[1]{\textcolor[rgb]{0.00,0.23,0.31}{#1}}
\newcommand{\DataTypeTok}[1]{\textcolor[rgb]{0.68,0.00,0.00}{#1}}
\newcommand{\DecValTok}[1]{\textcolor[rgb]{0.68,0.00,0.00}{#1}}
\newcommand{\DocumentationTok}[1]{\textcolor[rgb]{0.37,0.37,0.37}{\textit{#1}}}
\newcommand{\ErrorTok}[1]{\textcolor[rgb]{0.68,0.00,0.00}{#1}}
\newcommand{\ExtensionTok}[1]{\textcolor[rgb]{0.00,0.23,0.31}{#1}}
\newcommand{\FloatTok}[1]{\textcolor[rgb]{0.68,0.00,0.00}{#1}}
\newcommand{\FunctionTok}[1]{\textcolor[rgb]{0.28,0.35,0.67}{#1}}
\newcommand{\ImportTok}[1]{\textcolor[rgb]{0.00,0.46,0.62}{#1}}
\newcommand{\InformationTok}[1]{\textcolor[rgb]{0.37,0.37,0.37}{#1}}
\newcommand{\KeywordTok}[1]{\textcolor[rgb]{0.00,0.23,0.31}{#1}}
\newcommand{\NormalTok}[1]{\textcolor[rgb]{0.00,0.23,0.31}{#1}}
\newcommand{\OperatorTok}[1]{\textcolor[rgb]{0.37,0.37,0.37}{#1}}
\newcommand{\OtherTok}[1]{\textcolor[rgb]{0.00,0.23,0.31}{#1}}
\newcommand{\PreprocessorTok}[1]{\textcolor[rgb]{0.68,0.00,0.00}{#1}}
\newcommand{\RegionMarkerTok}[1]{\textcolor[rgb]{0.00,0.23,0.31}{#1}}
\newcommand{\SpecialCharTok}[1]{\textcolor[rgb]{0.37,0.37,0.37}{#1}}
\newcommand{\SpecialStringTok}[1]{\textcolor[rgb]{0.13,0.47,0.30}{#1}}
\newcommand{\StringTok}[1]{\textcolor[rgb]{0.13,0.47,0.30}{#1}}
\newcommand{\VariableTok}[1]{\textcolor[rgb]{0.07,0.07,0.07}{#1}}
\newcommand{\VerbatimStringTok}[1]{\textcolor[rgb]{0.13,0.47,0.30}{#1}}
\newcommand{\WarningTok}[1]{\textcolor[rgb]{0.37,0.37,0.37}{\textit{#1}}}

\providecommand{\tightlist}{%
  \setlength{\itemsep}{0pt}\setlength{\parskip}{0pt}}\usepackage{longtable,booktabs,array}
\usepackage{calc} % for calculating minipage widths
% Correct order of tables after \paragraph or \subparagraph
\usepackage{etoolbox}
\makeatletter
\patchcmd\longtable{\par}{\if@noskipsec\mbox{}\fi\par}{}{}
\makeatother
% Allow footnotes in longtable head/foot
\IfFileExists{footnotehyper.sty}{\usepackage{footnotehyper}}{\usepackage{footnote}}
\makesavenoteenv{longtable}
\usepackage{graphicx}
\makeatletter
\def\maxwidth{\ifdim\Gin@nat@width>\linewidth\linewidth\else\Gin@nat@width\fi}
\def\maxheight{\ifdim\Gin@nat@height>\textheight\textheight\else\Gin@nat@height\fi}
\makeatother
% Scale images if necessary, so that they will not overflow the page
% margins by default, and it is still possible to overwrite the defaults
% using explicit options in \includegraphics[width, height, ...]{}
\setkeys{Gin}{width=\maxwidth,height=\maxheight,keepaspectratio}
% Set default figure placement to htbp
\makeatletter
\def\fps@figure{htbp}
\makeatother

\KOMAoption{captions}{tableheading}
\makeatletter
\@ifpackageloaded{tcolorbox}{}{\usepackage[many]{tcolorbox}}
\@ifpackageloaded{fontawesome5}{}{\usepackage{fontawesome5}}
\definecolor{quarto-callout-color}{HTML}{909090}
\definecolor{quarto-callout-note-color}{HTML}{0758E5}
\definecolor{quarto-callout-important-color}{HTML}{CC1914}
\definecolor{quarto-callout-warning-color}{HTML}{EB9113}
\definecolor{quarto-callout-tip-color}{HTML}{00A047}
\definecolor{quarto-callout-caution-color}{HTML}{FC5300}
\definecolor{quarto-callout-color-frame}{HTML}{acacac}
\definecolor{quarto-callout-note-color-frame}{HTML}{4582ec}
\definecolor{quarto-callout-important-color-frame}{HTML}{d9534f}
\definecolor{quarto-callout-warning-color-frame}{HTML}{f0ad4e}
\definecolor{quarto-callout-tip-color-frame}{HTML}{02b875}
\definecolor{quarto-callout-caution-color-frame}{HTML}{fd7e14}
\makeatother
\makeatletter
\makeatother
\makeatletter
\makeatother
\makeatletter
\@ifpackageloaded{caption}{}{\usepackage{caption}}
\AtBeginDocument{%
\ifdefined\contentsname
  \renewcommand*\contentsname{Table of contents}
\else
  \newcommand\contentsname{Table of contents}
\fi
\ifdefined\listfigurename
  \renewcommand*\listfigurename{List of Figures}
\else
  \newcommand\listfigurename{List of Figures}
\fi
\ifdefined\listtablename
  \renewcommand*\listtablename{List of Tables}
\else
  \newcommand\listtablename{List of Tables}
\fi
\ifdefined\figurename
  \renewcommand*\figurename{Figure}
\else
  \newcommand\figurename{Figure}
\fi
\ifdefined\tablename
  \renewcommand*\tablename{Table}
\else
  \newcommand\tablename{Table}
\fi
}
\@ifpackageloaded{float}{}{\usepackage{float}}
\floatstyle{ruled}
\@ifundefined{c@chapter}{\newfloat{codelisting}{h}{lop}}{\newfloat{codelisting}{h}{lop}[chapter]}
\floatname{codelisting}{Listing}
\newcommand*\listoflistings{\listof{codelisting}{List of Listings}}
\makeatother
\makeatletter
\@ifpackageloaded{caption}{}{\usepackage{caption}}
\@ifpackageloaded{subcaption}{}{\usepackage{subcaption}}
\makeatother
\makeatletter
\@ifpackageloaded{tcolorbox}{}{\usepackage[many]{tcolorbox}}
\makeatother
\makeatletter
\@ifundefined{shadecolor}{\definecolor{shadecolor}{rgb}{.97, .97, .97}}
\makeatother
\makeatletter
\makeatother
\ifLuaTeX
  \usepackage{selnolig}  % disable illegal ligatures
\fi
\IfFileExists{bookmark.sty}{\usepackage{bookmark}}{\usepackage{hyperref}}
\IfFileExists{xurl.sty}{\usepackage{xurl}}{} % add URL line breaks if available
\urlstyle{same} % disable monospaced font for URLs
\hypersetup{
  pdftitle={HW \#1},
  pdfauthor={Vedika Shirtekar},
  colorlinks=true,
  linkcolor={blue},
  filecolor={Maroon},
  citecolor={Blue},
  urlcolor={Blue},
  pdfcreator={LaTeX via pandoc}}

\title{HW \#1}
\usepackage{etoolbox}
\makeatletter
\providecommand{\subtitle}[1]{% add subtitle to \maketitle
  \apptocmd{\@title}{\par {\large #1 \par}}{}{}
}
\makeatother
\subtitle{Interpreting \texttt{\{ggplot2\}} code}
\author{Vedika Shirtekar}
\date{}

\begin{document}
\maketitle
\ifdefined\Shaded\renewenvironment{Shaded}{\begin{tcolorbox}[interior hidden, breakable, boxrule=0pt, frame hidden, sharp corners, enhanced, borderline west={3pt}{0pt}{shadecolor}]}{\end{tcolorbox}}\fi

\renewcommand*\contentsname{Table of contents}
{
\hypersetup{linkcolor=}
\setcounter{tocdepth}{3}
\tableofcontents
}
\begin{tcolorbox}[enhanced jigsaw, colbacktitle=quarto-callout-tip-color!10!white, colframe=quarto-callout-tip-color-frame, arc=.35mm, titlerule=0mm, left=2mm, bottomtitle=1mm, opacityback=0, toptitle=1mm, title=\textcolor{quarto-callout-tip-color}{\faLightbulb}\hspace{0.5em}{Some notes before you get started}, leftrule=.75mm, colback=white, coltitle=black, opacitybacktitle=0.6, rightrule=.15mm, bottomrule=.15mm, toprule=.15mm, breakable]

\begin{itemize}
\tightlist
\item
  \textbf{Be sure to install any packages} in the Setup chunk that you
  don't already have.
\item
  \textbf{Leave the code chunk options, \texttt{eval:\ false} and
  \texttt{echo:\ true}, set as they are.} The final infographic has been
  intentionally optimized (e.g., text size, spacing) for saving and
  viewing as a PNG file, not for display in the Plots pane or within a
  rendered Quarto document. As a result, the text in each individual
  ggplot may appear too large when viewed in the Plots pane, but will be
  correctly sized in the exported PNG. We'll talk more about the nuances
  of saving ggplots (and why these differences occur) in a later lab
  section.
\item
  Some answers may become clearer once you've looked ahead at the code
  further down in the script. \textbf{Consider revisiting questions as
  you go.}
\end{itemize}

\end{tcolorbox}

\textbf{GitHub Repository Link:}
\url{https://github.com/vedikaS-byte/Eds-240-hw1-ufo.git}

\hypertarget{i.-setup}{%
\subsection{I. Setup}\label{i.-setup}}

\begin{Shaded}
\begin{Highlighting}[numbers=left,,]
\FunctionTok{library}\NormalTok{(colorspace)}
\FunctionTok{library}\NormalTok{(geofacet) }
\FunctionTok{library}\NormalTok{(ggtext) }
\FunctionTok{library}\NormalTok{(glue) }
\FunctionTok{library}\NormalTok{(grid)}
\FunctionTok{library}\NormalTok{(magick)}
\FunctionTok{library}\NormalTok{(patchwork) }
\FunctionTok{library}\NormalTok{(scales) }
\FunctionTok{library}\NormalTok{(showtext) }
\FunctionTok{library}\NormalTok{(tidyverse) }

\NormalTok{ufo\_sightings }\OtherTok{\textless{}{-}} \FunctionTok{read\_csv}\NormalTok{(}\StringTok{\textquotesingle{}https://raw.githubusercontent.com/rfordatascience/tidytuesday/main/data/2023/2023{-}06{-}20/ufo\_sightings.csv\textquotesingle{}}\NormalTok{)}
\NormalTok{places }\OtherTok{\textless{}{-}} \FunctionTok{read\_csv}\NormalTok{(}\StringTok{\textquotesingle{}https://raw.githubusercontent.com/rfordatascience/tidytuesday/main/data/2023/2023{-}06{-}20/places.csv\textquotesingle{}}\NormalTok{)}

\NormalTok{alien }\OtherTok{\textless{}{-}} \FunctionTok{c}\NormalTok{(}\StringTok{"\#101319"}\NormalTok{, }\StringTok{"\#28ee85"}\NormalTok{)}
\NormalTok{bg }\OtherTok{\textless{}{-}}\NormalTok{ alien[}\DecValTok{1}\NormalTok{]}
\NormalTok{accent }\OtherTok{\textless{}{-}}\NormalTok{ alien[}\DecValTok{2}\NormalTok{]}

\NormalTok{ufo\_image }\OtherTok{\textless{}{-}}\NormalTok{ magick}\SpecialCharTok{::}\FunctionTok{image\_read}\NormalTok{(}\AttributeTok{path =}\NormalTok{ here}\SpecialCharTok{::}\FunctionTok{here}\NormalTok{(}\StringTok{"MEDS"}\NormalTok{, }\StringTok{"eds{-}240"}\NormalTok{,}\StringTok{"Eds{-}240{-}hw1{-}ufo"}\NormalTok{ , }\StringTok{"images"}\NormalTok{, }\StringTok{"ufo.png"}\NormalTok{)) }

\NormalTok{sysfonts}\SpecialCharTok{::}\FunctionTok{font\_add\_google}\NormalTok{(}\AttributeTok{name =} \StringTok{"Orbitron"}\NormalTok{, }\AttributeTok{family =} \StringTok{"orb"}\NormalTok{)}
\NormalTok{sysfonts}\SpecialCharTok{::}\FunctionTok{font\_add\_google}\NormalTok{(}\AttributeTok{name =} \StringTok{"Barlow"}\NormalTok{, }\AttributeTok{family =} \StringTok{"bar"}\NormalTok{)}

\NormalTok{sysfonts}\SpecialCharTok{::}\FunctionTok{font\_add}\NormalTok{(}\AttributeTok{family =} \StringTok{"fa{-}brands"}\NormalTok{, }\AttributeTok{regular =}\NormalTok{ here}\SpecialCharTok{::}\FunctionTok{here}\NormalTok{(}\StringTok{"MEDS"}\NormalTok{, }\StringTok{"eds{-}240"}\NormalTok{,}\StringTok{"Eds{-}240{-}hw1{-}ufo"}\NormalTok{,}\StringTok{"fonts"}\NormalTok{, }\StringTok{"Font Awesome 6 Free{-}Regular{-}400.otf"}\NormalTok{))}
\NormalTok{sysfonts}\SpecialCharTok{::}\FunctionTok{font\_add}\NormalTok{(}\AttributeTok{family =} \StringTok{"fa{-}solid"}\NormalTok{, }\AttributeTok{regular =}\NormalTok{ here}\SpecialCharTok{::}\FunctionTok{here}\NormalTok{(}\StringTok{"MEDS"}\NormalTok{, }\StringTok{"eds{-}240"}\NormalTok{,}\StringTok{"Eds{-}240{-}hw1{-}ufo"}\NormalTok{, }\StringTok{"fonts"}\NormalTok{, }\StringTok{"Font Awesome 6 Free{-}Solid{-}900.otf"}\NormalTok{))}

\NormalTok{showtext}\SpecialCharTok{::}\FunctionTok{showtext\_auto}\NormalTok{(}\AttributeTok{enable =} \ConstantTok{TRUE}\NormalTok{)}
\end{Highlighting}
\end{Shaded}

\begin{enumerate}
\def\labelenumi{\arabic{enumi}.}
\tightlist
\item
  \textbf{What is the author defining in lines 15-17? Where else in the
  code do these defined variables show up? What advantage(s) is there to
  defining these values here, as variables, rather than defining the
  values directly throughout the script?}

  \begin{itemize}
  \item
    In lines 15-17, a color palette is defined with hex codes for
    background (black) and accent (green) colors in the UFO visual. Each
    color is subset by index and assigned to a variable that is
    referenced for a specific use. For example, \texttt{bg} contains the
    black background color for the visual, which is referred to in the
    aesthetics for constructing the plot background fill and base color,
    such as in \texttt{plot\_us} and the text color for
    \texttt{plot\_shape}. The \texttt{accent} variable refers to the
    light green color primarily used for text color and the fill of
    graph aesthetics, such as bar fills (ex. \texttt{accent} is used to
    define the text color and fill color of the graphs in
    \texttt{plot\_us} and \texttt{plot\_day}).
  \item
    The naming convention is convenient because saving the colors as
    variables prevents having to constantly reference the hex codes
    throughout the script. Additionally, it makes the code easier to
    read, maintain, and update, since changing a color only requires
    editing it in one place rather than multiple locations.
  \end{itemize}
\item
  \textbf{In your own words, explain what the function,
  \texttt{font\_add\_google()}, does. What's the difference between the
  two arguments, \texttt{name} and \texttt{family}?}
\end{enumerate}

\begin{itemize}
\tightlist
\item
  \texttt{font\_add\_google()} is a function from the \texttt{sysfonts}
  package that downloads font files from the Google Fonts repository and
  registers them so they can be used in R graphics. The function
  searches for a font using its official Google Fonts name, downloads
  the necessary files, and makes the font available for plotting. The
  \texttt{name} argument specifies the exact Google Fonts name used to
  identify the preferred font. The \texttt{family} argument defines the
  font family name used within R, which acts as an internal label
  referenced by plotting functions, such as ggplot2. While
  \texttt{family} can be any string and does not need to match
  \texttt{name}, it is the value that can be later used when specifying
  the font in plots.
\end{itemize}

\hypertarget{ii.-data-wrangling}{%
\subsection{II. Data wrangling}\label{ii.-data-wrangling}}

\hypertarget{i.-create-df_pop}{%
\subsubsection{\texorpdfstring{i. Create
\texttt{df\_pop}}{i. Create df\_pop}}\label{i.-create-df_pop}}

\begin{Shaded}
\begin{Highlighting}[numbers=left,,]
\NormalTok{df\_pop }\OtherTok{\textless{}{-}}\NormalTok{ places }\SpecialCharTok{|\textgreater{}}
  \FunctionTok{filter}\NormalTok{(country\_code }\SpecialCharTok{==} \StringTok{"US"}\NormalTok{) }\SpecialCharTok{|\textgreater{}}
  \FunctionTok{mutate}\NormalTok{(}\AttributeTok{state =} \FunctionTok{str\_replace}\NormalTok{(}\AttributeTok{string =}\NormalTok{ state,}
                             \AttributeTok{pattern =} \StringTok{"Fl"}\NormalTok{,}
                             \AttributeTok{replacement =} \StringTok{"FL"}\NormalTok{)) }\SpecialCharTok{|\textgreater{}} 
  \FunctionTok{group\_by}\NormalTok{(state) }\SpecialCharTok{|\textgreater{}}
  \FunctionTok{summarise}\NormalTok{(}\AttributeTok{pop =} \FunctionTok{sum}\NormalTok{(population)) }\SpecialCharTok{|\textgreater{}}
  \FunctionTok{ungroup}\NormalTok{()}
\end{Highlighting}
\end{Shaded}

\begin{enumerate}
\def\labelenumi{\arabic{enumi}.}
\setcounter{enumi}{2}
\item
  \textbf{Describe what this data frame contains.}

  \begin{itemize}
  \tightlist
  \item
    This data frame contains aggregated population data for the United
    States at the state level (grouped by state). The \texttt{df\_pop}
    data set is created by filtering the \texttt{places} data set to
    include only records from the U.S., setting the state abbreviation
    for Florida, and then grouping the data by state. For each state,
    the population values of all places are summed to calculate the
    total state population. The finalized data frame has one observation
    per state with a column for the \texttt{state} abbreviation and a
    column for the total population (\texttt{pop}).
  \end{itemize}
\end{enumerate}

\hypertarget{ii.-create-df_us}{%
\subsubsection{\texorpdfstring{ii. Create
\texttt{df\_us}}{ii. Create df\_us}}\label{ii.-create-df_us}}

\begin{Shaded}
\begin{Highlighting}[numbers=left,,]
\NormalTok{df\_us }\OtherTok{\textless{}{-}}\NormalTok{ ufo\_sightings }\SpecialCharTok{|\textgreater{}}
  \FunctionTok{filter}\NormalTok{(country\_code }\SpecialCharTok{==} \StringTok{"US"}\NormalTok{) }\SpecialCharTok{|\textgreater{}}
  \FunctionTok{mutate}\NormalTok{(}\AttributeTok{state =} \FunctionTok{str\_replace}\NormalTok{(}\AttributeTok{string =}\NormalTok{ state,}
                             \AttributeTok{pattern =} \StringTok{"Fl"}\NormalTok{,}
                             \AttributeTok{replacement =} \StringTok{"FL"}\NormalTok{)) }\SpecialCharTok{|\textgreater{}} 
  \FunctionTok{count}\NormalTok{(state) }\SpecialCharTok{|\textgreater{}}
  \FunctionTok{left\_join}\NormalTok{(df\_pop, }\AttributeTok{by =} \StringTok{"state"}\NormalTok{) }\SpecialCharTok{|\textgreater{}}
  \FunctionTok{rename}\NormalTok{(}\AttributeTok{num\_obs =}\NormalTok{ n) }\SpecialCharTok{|\textgreater{}} 
  \FunctionTok{mutate}\NormalTok{(}
    \AttributeTok{num\_obs\_per10k =}\NormalTok{ num\_obs }\SpecialCharTok{/}\NormalTok{ pop }\SpecialCharTok{*} \DecValTok{10000}\NormalTok{,}
    \AttributeTok{opacity\_val =}\NormalTok{ num\_obs\_per10k }\SpecialCharTok{/} \FunctionTok{max}\NormalTok{(num\_obs\_per10k)}
\NormalTok{    )}
\end{Highlighting}
\end{Shaded}

\begin{enumerate}
\def\labelenumi{\arabic{enumi}.}
\setcounter{enumi}{3}
\item
  \textbf{Describe what this data frame contains.}

  \begin{itemize}
  \tightlist
  \item
    The \texttt{df\_us} data frame contains the summarized UFO sighting
    information at the state level (including Washington DC). The data
    frame includes the total number of reported UFO sightings
    (\texttt{num\_obs}), the state's total population (\texttt{pop}),
    and a normalized rate of sightings per 10,000 people
    (\texttt{num\_obs\_per10k}). \texttt{opacity\_val} contains the
    scaled sightings rate relative to the maximum observed rate of
    sightings per state.
  \end{itemize}
\item
  \textbf{What does \texttt{opacity\_val} represent, and why is it
  calculated?}

  \begin{itemize}
  \tightlist
  \item
    \texttt{opacity\_val} represents the scaled measure of UFO sightings
    rate or frequency relative to the maximum observed rate of sightings
    per state with a range between 0 and 1. \texttt{opacity\_val} is
    calculated by dividing each state's sightings per 10,000 people
    (\texttt{num\_obs\_per10k}) by the maximum value of
    \texttt{num\_obs\_per10k} across all states. This variable was
    calculated for visualization purposes, such as mapping
    \texttt{opacity\_val} to transparency (\texttt{alpha}) in a plot or
    map. With this aesthetic, states with a higher rate of UFO sightings
    would appear lighter, while states with lower rates would appear
    darker. By mapping \texttt{opacity\_val} to adjust for transparency
    in a ggplot object, it is easier to visually compare relative
    sighting intensity across states.
  \end{itemize}
\end{enumerate}

\hypertarget{iii.-create-df_shape}{%
\subsubsection{\texorpdfstring{iii. Create
\texttt{df\_shape}}{iii. Create df\_shape}}\label{iii.-create-df_shape}}

\begin{Shaded}
\begin{Highlighting}[numbers=left,,]
\NormalTok{df\_shape }\OtherTok{\textless{}{-}}\NormalTok{ ufo\_sightings }\SpecialCharTok{|\textgreater{}}
  \FunctionTok{filter}\NormalTok{(}\SpecialCharTok{!}\NormalTok{shape }\SpecialCharTok{\%in\%} \FunctionTok{c}\NormalTok{(}\StringTok{"unknown"}\NormalTok{, }\StringTok{"other"}\NormalTok{)) }\SpecialCharTok{|\textgreater{}}
  \FunctionTok{count}\NormalTok{(shape) }\SpecialCharTok{|\textgreater{}}
  \FunctionTok{rename}\NormalTok{(}\AttributeTok{total\_sightings =}\NormalTok{ n) }\SpecialCharTok{|\textgreater{}} 
  \FunctionTok{arrange}\NormalTok{(}\FunctionTok{desc}\NormalTok{(total\_sightings)) }\SpecialCharTok{|\textgreater{}}
  \FunctionTok{slice\_head}\NormalTok{(}\AttributeTok{n =} \DecValTok{10}\NormalTok{) }\SpecialCharTok{|\textgreater{}}
  \FunctionTok{mutate}\NormalTok{(}
    \AttributeTok{shape =} \FunctionTok{fct\_reorder}\NormalTok{(}\AttributeTok{.f =}\NormalTok{ shape, }
                        \AttributeTok{.x =}\NormalTok{ total\_sightings), }
    \AttributeTok{opacity\_val =}\NormalTok{ scales}\SpecialCharTok{::}\FunctionTok{rescale}\NormalTok{(}\AttributeTok{x =}\NormalTok{ total\_sightings, }
                                  \AttributeTok{to =} \FunctionTok{c}\NormalTok{(}\FloatTok{0.3}\NormalTok{, }\DecValTok{1}\NormalTok{))}
\NormalTok{    )}
\end{Highlighting}
\end{Shaded}

\begin{enumerate}
\def\labelenumi{\arabic{enumi}.}
\setcounter{enumi}{5}
\item
  \textbf{Describe what this data frame contains.}

  \begin{itemize}
  \tightlist
  \item
    \texttt{df\_shape} contains the total sightings of each differently
    shaped UFO, summarizing the top ten most commonly reported UFO
    shapes. The number of total sightings are arranged in descending
    order, and vague shapes (eg., ``unknown and''other'') are excluded
    from the data frame. \texttt{opacity\_val} rescales the total number
    of sightings to a range between 0.3 and 1.
  \end{itemize}
\item
  \textbf{What does \texttt{fct\_reorder} do when it is applied to the
  \texttt{shape} variable? What would have happened if this step was not
  performed?}

  \begin{itemize}
  \tightlist
  \item
    \texttt{fct\_reorder()} reorders the levels of the \texttt{shape}
    factor based on the values of \texttt{total\_sightings}. The UFO
    shape categories are arranged in order of decreasing frequency
    rather than remaining in the default or alphabetical order, ensuring
    that when plotted, the shapes appear in a meaningful order that
    reflects how common each shape is (ex. plotting each level in
    descending order rather than alphabetically). Without reordering the
    levels in the \texttt{shape} factor, the \texttt{shape} variable
    would remain in its original ordering, and the plot using
    \texttt{shape} on its respective access would display the categories
    in a nonsensical numerical order; it would be harder to visually
    interpret and compare the frequency of different UFO shapes.
  \end{itemize}
\item
  \textbf{What is the purpose of rescaling \texttt{opacity\_val}? And
  why rescale from 0.3 to 1?}

  \begin{itemize}
  \tightlist
  \item
    \texttt{opacity\_val} was rescaled to convert the actual counts of
    UFO sightings into values that can be utilized directly for
    visualization. More specifically, opacity\_val can be utilized in
    the \texttt{alpha} argument of \texttt{ggplot} functions to
    represent the scale of ``emphasis'' on overall sightings per UFO
    shape type. Rescaling from 0.3 to 1 ensures that all shapes remain
    visible while still highlighting differences in frequency. For
    example, if the lower bound were 0 or close to 0, then shapes with
    fewer sightings (or very close to it), would become too dark to spot
    in a plot. By setting the minimum opacity to 0.3, the least common
    shapes are still easy to see. The upper bound of 1 allows the most
    frequently reported shapes to be fully colored and stand out more
    clearly. making them stand out clearly. This range strikes a balance
    between readability and visual emphasis in the visualization.
  \end{itemize}
\end{enumerate}

\hypertarget{iv.-create-df_day_hour}{%
\subsubsection{\texorpdfstring{iv. Create
\texttt{df\_day\_hour}}{iv. Create df\_day\_hour}}\label{iv.-create-df_day_hour}}

\begin{Shaded}
\begin{Highlighting}[numbers=left,,]
\NormalTok{df\_day\_hour }\OtherTok{\textless{}{-}}\NormalTok{ ufo\_sightings }\SpecialCharTok{|\textgreater{}}
  \FunctionTok{mutate}\NormalTok{(}
    \AttributeTok{day =} \FunctionTok{wday}\NormalTok{(reported\_date\_time), }
    \AttributeTok{hour =} \FunctionTok{hour}\NormalTok{(reported\_date\_time), }
    \AttributeTok{wday =} \FunctionTok{wday}\NormalTok{(reported\_date\_time, }\AttributeTok{label =} \ConstantTok{TRUE}\NormalTok{) }
\NormalTok{  ) }\SpecialCharTok{|\textgreater{}}
  \FunctionTok{count}\NormalTok{(day, wday, hour) }\SpecialCharTok{|\textgreater{}}
  \FunctionTok{rename}\NormalTok{(}\AttributeTok{total\_daily\_obs =}\NormalTok{ n) }\SpecialCharTok{|\textgreater{}} 
  \FunctionTok{mutate}\NormalTok{(}
    \AttributeTok{opacity\_val =}\NormalTok{ total\_daily\_obs }\SpecialCharTok{/} \FunctionTok{max}\NormalTok{(total\_daily\_obs),}
    \AttributeTok{hour\_lab =} \FunctionTok{case\_when}\NormalTok{(}
\NormalTok{      hour }\SpecialCharTok{==} \DecValTok{0} \SpecialCharTok{\textasciitilde{}} \StringTok{"12am"}\NormalTok{,}
\NormalTok{      hour }\SpecialCharTok{\textless{}=} \DecValTok{12} \SpecialCharTok{\textasciitilde{}} \FunctionTok{paste0}\NormalTok{(hour, }\StringTok{"am"}\NormalTok{),}
\NormalTok{      hour }\SpecialCharTok{==} \DecValTok{12} \SpecialCharTok{\textasciitilde{}} \StringTok{"12pm"}\NormalTok{,}
      \ConstantTok{TRUE} \SpecialCharTok{\textasciitilde{}} \FunctionTok{paste0}\NormalTok{(hour }\SpecialCharTok{{-}} \DecValTok{12}\NormalTok{, }\StringTok{"pm"}\NormalTok{)) }
\NormalTok{    )}
\end{Highlighting}
\end{Shaded}

\begin{enumerate}
\def\labelenumi{\arabic{enumi}.}
\setcounter{enumi}{8}
\item
  \textbf{Describe what this data frame contains.}

  \begin{itemize}
  \tightlist
  \item
    \texttt{df\_day\_hour} contains the total daily observations and
    scaled opacity for UFO sightings by day of the week and hour of the
    day. For each combination of day and hour, \texttt{df\_day\_hour}
    records the total number of sightings (\texttt{total\_daily\_obs})
    and a scaled value (\texttt{opacity\_val}) that indicates the
    relative intensity of sightings for visualization purposes.
  \end{itemize}
\item
  \textbf{What is the purpose of the last line inside the
  \texttt{case\_when()} statement
  (\texttt{TRUE\ \textasciitilde{}\ paste0(hour\ -\ 12,\ "pm")})?}

  \begin{itemize}
  \tightlist
  \item
    The last line in the \texttt{case\_when()} statement
    \texttt{(TRUE\ \textasciitilde{}\ paste0(hour\ -\ 12,\ "pm"))}
    converts hours greater than 12 from 24-hour time to 12-hour PM
    format such that every hour has a readable label
    (\texttt{hour\_lab}).
  \end{itemize}
\end{enumerate}

\hypertarget{iii.-prepare-text-elements}{%
\subsection{III. Prepare text
elements}\label{iii.-prepare-text-elements}}

\begin{Shaded}
\begin{Highlighting}[numbers=left,,]
\NormalTok{quotes }\OtherTok{\textless{}{-}} \FunctionTok{paste0}\NormalTok{(}\StringTok{\textquotesingle{}"...\textquotesingle{}}\NormalTok{, }\FunctionTok{str\_to\_sentence}\NormalTok{(ufo\_sightings}\SpecialCharTok{$}\NormalTok{summary[}\FunctionTok{c}\NormalTok{(}\DecValTok{47816}\NormalTok{, }\DecValTok{6795}\NormalTok{, }\DecValTok{93833}\NormalTok{)]), }\StringTok{\textquotesingle{}..."\textquotesingle{}}\NormalTok{)}

\NormalTok{original }\OtherTok{\textless{}{-}} \FunctionTok{glue}\NormalTok{(}\StringTok{"Original visualization by Dan Oehm:"}\NormalTok{)}
\NormalTok{dan\_github }\OtherTok{\textless{}{-}} \FunctionTok{glue}\NormalTok{(}\StringTok{"\textless{}span style=\textquotesingle{}font{-}family:fa{-}brands;\textquotesingle{}\textgreater{}\&\#xf09b;\textless{}/span\textgreater{} doehm/tidytues"}\NormalTok{)}
\NormalTok{new }\OtherTok{\textless{}{-}} \FunctionTok{glue}\NormalTok{(}\StringTok{"Updated version by Sam Shanny{-}Csik for EDS 240:"}\NormalTok{)}
\NormalTok{link }\OtherTok{\textless{}{-}} \FunctionTok{glue}\NormalTok{(}\StringTok{"\textless{}span style=\textquotesingle{}font{-}family:fa{-}solid;\textquotesingle{}\textgreater{}\&\#xf0c1;\textless{}/span\textgreater{} eds{-}240{-}data{-}viz.github.io"}\NormalTok{)}
\NormalTok{space }\OtherTok{\textless{}{-}} \FunctionTok{glue}\NormalTok{(}\StringTok{"\textless{}span style=\textquotesingle{}color:\{bg\};\textquotesingle{}\textgreater{}. .\textless{}/span\textgreater{}"}\NormalTok{)}
\NormalTok{caption }\OtherTok{\textless{}{-}} \FunctionTok{glue}\NormalTok{(}\StringTok{"\{original\}\{space\}\{dan\_github\}}
\StringTok{                \textless{}br\textgreater{}\textless{}br\textgreater{}}
\StringTok{                \{new\}\{space\}\{link\}"}\NormalTok{)}
\end{Highlighting}
\end{Shaded}

\begin{enumerate}
\def\labelenumi{\arabic{enumi}.}
\setcounter{enumi}{10}
\item
  \textbf{In your own words, what is the difference between
  \texttt{paste0()} and \texttt{glue()}? Why did the author use
  \texttt{paste0} to construct \texttt{quotes} and \texttt{glue} to
  construct the other text elements?}

  \begin{itemize}
  \tightlist
  \item
    The difference between \texttt{paste0()} and \texttt{glue()} is
    defined in how both functions handle combining text with variables.
    \texttt{paste0()} concatenates strings and values without
    considering variable names inside the string; this is suitable for
    straightforwardly combining fixed text and data. As observed in
    \texttt{quotes}, the author utilized \texttt{paste0()} to add a
    prefix and suffix of ``\ldots{}'' around the selected summary
    sentences in \texttt{ufo\_sightings}. \texttt{glue()} allows the
    author to write a string with variable placeholders that are
    automatically replaced with their respective values, allowing it to
    be cleaner and easier to use for formatted text. The author used
    \texttt{paste0()} for \texttt{quotes} because the task was simple
    concatenation (strings tied to a subsetted part of the data frame),
    while \texttt{glue()} was used for the other elements to make the
    code cleaner and easier to read when building formatted strings
    (using \texttt{\{variable\}}).
  \end{itemize}
\end{enumerate}

\hypertarget{iv.-build-plots}{%
\subsection{IV. Build plots}\label{iv.-build-plots}}

\hypertarget{i.-build-plot_shape}{%
\subsubsection{\texorpdfstring{i. Build
\texttt{plot\_shape}}{i. Build plot\_shape}}\label{i.-build-plot_shape}}

\begin{Shaded}
\begin{Highlighting}[numbers=left,,]
\NormalTok{plot\_shape }\OtherTok{\textless{}{-}} \FunctionTok{ggplot}\NormalTok{(}\AttributeTok{data =}\NormalTok{ df\_shape) }\SpecialCharTok{+}
  \FunctionTok{geom\_col}\NormalTok{(}\FunctionTok{aes}\NormalTok{(}\AttributeTok{x =}\NormalTok{ total\_sightings, }\AttributeTok{y =}\NormalTok{ shape, }\AttributeTok{alpha =}\NormalTok{ opacity\_val), }
           \AttributeTok{fill =}\NormalTok{ accent) }\SpecialCharTok{+}
  \FunctionTok{geom\_text}\NormalTok{(}\FunctionTok{aes}\NormalTok{(}\AttributeTok{x =} \DecValTok{200}\NormalTok{, }\AttributeTok{y =}\NormalTok{ shape, }\AttributeTok{label =} \FunctionTok{str\_to\_title}\NormalTok{(shape)), }
            \AttributeTok{family =} \StringTok{"orb"}\NormalTok{, }
            \AttributeTok{fontface =} \StringTok{"bold"}\NormalTok{,}
            \AttributeTok{color =}\NormalTok{ bg, }
            \AttributeTok{size =} \DecValTok{14}\NormalTok{, }
            \AttributeTok{hjust =} \DecValTok{0}\NormalTok{,}
            \AttributeTok{nudge\_y =} \FloatTok{0.2}\NormalTok{) }\SpecialCharTok{+}
  \FunctionTok{geom\_text}\NormalTok{(}\FunctionTok{aes}\NormalTok{(}\AttributeTok{x =}\NormalTok{ total\_sightings}\DecValTok{{-}200}\NormalTok{, }\AttributeTok{y =}\NormalTok{ shape, }\AttributeTok{label =}\NormalTok{ scales}\SpecialCharTok{::}\FunctionTok{comma}\NormalTok{(total\_sightings)),}
            \AttributeTok{family =} \StringTok{"orb"}\NormalTok{,}
            \AttributeTok{fontface =} \StringTok{"bold"}\NormalTok{,}
            \AttributeTok{color =}\NormalTok{ bg,}
            \AttributeTok{size =} \DecValTok{10}\NormalTok{,}
            \AttributeTok{hjust =} \DecValTok{1}\NormalTok{,}
            \AttributeTok{nudge\_y =} \SpecialCharTok{{-}}\FloatTok{0.2}\NormalTok{) }\SpecialCharTok{+}
  \FunctionTok{scale\_x\_continuous}\NormalTok{(}\AttributeTok{expand =} \FunctionTok{c}\NormalTok{(}\DecValTok{0}\NormalTok{, }\ConstantTok{NA}\NormalTok{)) }\SpecialCharTok{+}
  \FunctionTok{labs}\NormalTok{(}\AttributeTok{subtitle =} \StringTok{"10 most commonly reported shapes"}\NormalTok{) }\SpecialCharTok{+}
  \FunctionTok{theme\_void}\NormalTok{() }\SpecialCharTok{+}
  \FunctionTok{theme}\NormalTok{(}
    \AttributeTok{plot.subtitle =} \FunctionTok{element\_text}\NormalTok{(}\AttributeTok{family =} \StringTok{"bar"}\NormalTok{, }
                                 \AttributeTok{size =} \DecValTok{40}\NormalTok{, }
                                 \AttributeTok{color =}\NormalTok{ accent,}
                                 \AttributeTok{hjust =} \DecValTok{0}\NormalTok{,  }
                                 \AttributeTok{margin =} \FunctionTok{margin}\NormalTok{(}\AttributeTok{b =} \DecValTok{10}\NormalTok{)),}
    \AttributeTok{legend.position =} \StringTok{"none"} 
\NormalTok{  )}
\end{Highlighting}
\end{Shaded}

\begin{enumerate}
\def\labelenumi{\arabic{enumi}.}
\setcounter{enumi}{11}
\item
  \textbf{Explain the values provided to the \texttt{x} aesthetic for
  both text geoms (\texttt{shape} \& \texttt{total\_sightings}).}

  \begin{itemize}
  \tightlist
  \item
    In \texttt{plot\_shape}, the x values for the two
    \texttt{geom\_text()} layers control the horizontal placement of the
    text relative to the bars. The first \texttt{geom\_text()} labels
    the \texttt{shape} names, and \texttt{x\ =\ 200} sets a fixed
    position on the x-axis such that all the shape names appear aligned
    to the left of the horizontal bars. In the second
    \texttt{geom\_text()}, the \texttt{x} aesthetic is assigned to
    display the total sightings per UFO shape
    (\texttt{total\_sightings}); \texttt{x\ =\ total\_sightings\ -\ 200}
    places the text for the number of total sightings at the end of each
    bar's length.
  \end{itemize}
\end{enumerate}

\hypertarget{ii.-build-plot_us}{%
\subsubsection{\texorpdfstring{ii. Build
\texttt{plot\_us}}{ii. Build plot\_us}}\label{ii.-build-plot_us}}

\textbf{HINT:} Consider temporarily commenting out / rearranging the
\texttt{geom\_*()} layers to better understand how this plot is
constructed

\begin{Shaded}
\begin{Highlighting}[numbers=left,,]
\NormalTok{plot\_us }\OtherTok{\textless{}{-}}  \FunctionTok{ggplot}\NormalTok{(df\_us) }\SpecialCharTok{+}
  \FunctionTok{geom\_rect}\NormalTok{(}\FunctionTok{aes}\NormalTok{(}\AttributeTok{xmin =} \DecValTok{0}\NormalTok{, }\AttributeTok{xmax =} \DecValTok{1}\NormalTok{, }\AttributeTok{ymin =} \DecValTok{0}\NormalTok{, }\AttributeTok{ymax =} \DecValTok{1}\NormalTok{, }\AttributeTok{alpha =}\NormalTok{ opacity\_val), }
            \AttributeTok{fill =}\NormalTok{ accent) }\SpecialCharTok{+}
  \FunctionTok{geom\_text}\NormalTok{(}\FunctionTok{aes}\NormalTok{(}\AttributeTok{x =} \FloatTok{0.5}\NormalTok{, }\AttributeTok{y =} \FloatTok{0.7}\NormalTok{, }\AttributeTok{label =}\NormalTok{ state), }
            \AttributeTok{family =} \StringTok{"orb"}\NormalTok{, }
            \AttributeTok{fontface =} \StringTok{"bold"}\NormalTok{,}
            \AttributeTok{size =} \DecValTok{9}\NormalTok{, }
            \AttributeTok{color =}\NormalTok{ bg) }\SpecialCharTok{+}
  \FunctionTok{geom\_text}\NormalTok{(}\FunctionTok{aes}\NormalTok{(}\AttributeTok{x =} \FloatTok{0.5}\NormalTok{, }\AttributeTok{y =} \FloatTok{0.3}\NormalTok{, }\AttributeTok{label =} \FunctionTok{round}\NormalTok{(num\_obs\_per10k, }\DecValTok{1}\NormalTok{)), }
            \AttributeTok{family =} \StringTok{"orb"}\NormalTok{, }
            \AttributeTok{fontface =} \StringTok{"bold"}\NormalTok{,}
            \AttributeTok{size =} \DecValTok{8}\NormalTok{, }
            \AttributeTok{color =}\NormalTok{ bg) }\SpecialCharTok{+}  
\NormalTok{  geofacet}\SpecialCharTok{::}\FunctionTok{facet\_geo}\NormalTok{(}\SpecialCharTok{\textasciitilde{}}\NormalTok{state) }\SpecialCharTok{+}
  \FunctionTok{coord\_fixed}\NormalTok{(}\AttributeTok{ratio =} \DecValTok{1}\NormalTok{) }\SpecialCharTok{+}
  \FunctionTok{labs}\NormalTok{(}\AttributeTok{subtitle =} \StringTok{"Sightings per 10k population"}\NormalTok{) }\SpecialCharTok{+}
  \FunctionTok{theme\_void}\NormalTok{() }\SpecialCharTok{+}
  \FunctionTok{theme}\NormalTok{(}
    \AttributeTok{strip.text =} \FunctionTok{element\_blank}\NormalTok{(),}
    \AttributeTok{plot.subtitle =} \FunctionTok{element\_text}\NormalTok{(}\AttributeTok{family =} \StringTok{"bar"}\NormalTok{, }
                                 \AttributeTok{size =} \DecValTok{40}\NormalTok{, }
                                 \AttributeTok{color =}\NormalTok{ accent,}
                                 \AttributeTok{hjust =} \DecValTok{1}\NormalTok{,  }
                                 \AttributeTok{margin =} \FunctionTok{margin}\NormalTok{(}\AttributeTok{b =} \DecValTok{10}\NormalTok{)),}
    \AttributeTok{legend.position =} \StringTok{"none"} 
\NormalTok{  )}
\end{Highlighting}
\end{Shaded}

\begin{enumerate}
\def\labelenumi{\arabic{enumi}.}
\setcounter{enumi}{12}
\item
  \textbf{Consider the order of \texttt{geom\_*()} layers in the the
  above plot (\texttt{plot\_us}). Why did the author order the layers in
  this way?}

  \begin{itemize}
  \tightlist
  \item
    In \texttt{plot\_us}, the order of the \texttt{geom\_*()} layers is
    intentional because ggplot draws layers in the order they are added,
    where earlier layers appear underneath later ones. The first layer
    is \texttt{geom\_rect()}, which created the colored rectangle for
    each facet to serve as a background representing the opacity of UFO
    sightings. By placing it first, the rectangles appear behind all
    proceeding elements. The state labels can then be added onto the
    plot and faceted by state, adding the state labels and the sightings
    per 10,000 state residents on top of the rectangles. By drawing the
    text after the background is established, the labels will be clearly
    visible and readable. If the order were reversed and the rectangles
    were drawn after the text, then the text would be hidden and
    difficult to interpret.
  \end{itemize}
\end{enumerate}

\hypertarget{iii.-build-plot_day}{%
\subsubsection{\texorpdfstring{iii. Build
\texttt{plot\_day}}{iii. Build plot\_day}}\label{iii.-build-plot_day}}

\begin{Shaded}
\begin{Highlighting}[numbers=left,,]
\NormalTok{plot\_day }\OtherTok{\textless{}{-}} \FunctionTok{ggplot}\NormalTok{(}\AttributeTok{data =}\NormalTok{ df\_day\_hour) }\SpecialCharTok{+}
  \FunctionTok{geom\_tile}\NormalTok{(}\FunctionTok{aes}\NormalTok{(}\AttributeTok{x =}\NormalTok{ hour, }\AttributeTok{y =}\NormalTok{ day, }\AttributeTok{alpha =}\NormalTok{ opacity\_val), }
            \AttributeTok{fill =}\NormalTok{ accent, }
            \AttributeTok{height =} \FloatTok{0.9}\NormalTok{, }
            \AttributeTok{width =} \FloatTok{0.9}\NormalTok{) }\SpecialCharTok{+}
  \FunctionTok{geom\_text}\NormalTok{(}\FunctionTok{aes}\NormalTok{(}\AttributeTok{x =}\NormalTok{ hour, }\AttributeTok{y =} \DecValTok{9}\NormalTok{, }\AttributeTok{label =}\NormalTok{ hour\_lab), }
            \AttributeTok{family =} \StringTok{"orb"}\NormalTok{,}
            \AttributeTok{color =}\NormalTok{ accent, }
            \AttributeTok{size =} \DecValTok{10}\NormalTok{) }\SpecialCharTok{+}
  \FunctionTok{geom\_text}\NormalTok{(}\FunctionTok{aes}\NormalTok{(}\AttributeTok{x =} \DecValTok{0}\NormalTok{, }\AttributeTok{y =}\NormalTok{ day, }\AttributeTok{label =} \FunctionTok{str\_sub}\NormalTok{(}\AttributeTok{string =}\NormalTok{ wday, }\AttributeTok{start =} \DecValTok{1}\NormalTok{, }\AttributeTok{end =} \DecValTok{1}\NormalTok{)), }
            \AttributeTok{family =} \StringTok{"orb"}\NormalTok{, }
            \AttributeTok{fontface =} \StringTok{"bold"}\NormalTok{,}
            \AttributeTok{color =}\NormalTok{ bg, }
            \AttributeTok{size =} \DecValTok{8}\NormalTok{) }\SpecialCharTok{+} 
  \FunctionTok{ylim}\NormalTok{(}\SpecialCharTok{{-}}\DecValTok{5}\NormalTok{, }\DecValTok{9}\NormalTok{) }\SpecialCharTok{+}
  \FunctionTok{xlim}\NormalTok{(}\ConstantTok{NA}\NormalTok{, }\FloatTok{23.55}\NormalTok{) }\SpecialCharTok{+}
  \FunctionTok{coord\_polar}\NormalTok{() }\SpecialCharTok{+}
  \FunctionTok{theme\_void}\NormalTok{() }\SpecialCharTok{+}
  \FunctionTok{theme}\NormalTok{(}
    \AttributeTok{plot.background =} \FunctionTok{element\_rect}\NormalTok{(}\AttributeTok{fill =}\NormalTok{ , }\AttributeTok{color =}\NormalTok{ bg),}
    \AttributeTok{legend.position =} \StringTok{"none"}
\NormalTok{  )}
\end{Highlighting}
\end{Shaded}

\begin{enumerate}
\def\labelenumi{\arabic{enumi}.}
\setcounter{enumi}{13}
\item
  \textbf{This plot includes one-letter labels for each day of the week.
  How is this accomplished when week days are written using their
  three-letter abbreviations (e.g.~\texttt{Mon}, \texttt{Tue}) in the
  \texttt{df\_day\_hour} data frame?}

  \begin{itemize}
  \tightlist
  \item
    The one-letter weekday labels are created by extracting the first
    letter of the three-letter weekday abbreviations in the
    \texttt{wday} column. This is done in \texttt{geom\_text()} with
    \texttt{label\ =\ str\_sub(string\ =\ wday,\ start\ =\ 1,\ end\ =\ 1)},
    which takes only the first character of each weekday (ex. ``M'' from
    ``Mon'', ``T'' from ``Tue'', etc.) to display as a single-letter
    label on the plot.
  \end{itemize}
\item
  \textbf{What role do the \texttt{ylim()} and \texttt{xlim()} functions
  play in shaping a ggplot, and how do they change the visual layout of
  this particular plot? To better understand their effect, try rerunning
  the code with each of these lines commented out and observe how the
  plot's spacing and composition change.}

  \begin{itemize}
  \tightlist
  \item
    In ggplot, the \texttt{xlim()} and \texttt{ylim()} functions set
    specific limits on the x and y axes, controlling the range of values
    that are displayed and influencing the spacing of plot elements. In
    \texttt{plot\_day}, the polar coordinate system is used, so limits
    are set for shaping a circular layout. The \texttt{ylim(-5,\ 9)}
    extends the y-axis beyond the range of the day labels, creating
    space for the one-letter weekday labels and hour labels such that
    they do not overlap with the tiles. \texttt{xlim(NA,\ 23.55)}
    slightly extends the x-axis beyond the maximum hour (12 AM) to
    ensure that the 12 AM hour tile and its label are visible. Without
    these limits, the axes auto scale to the data frame, causing text
    labels to overlap or collide with the tile; the overall circular
    layout becomes cramped.
  \end{itemize}
\end{enumerate}

\hypertarget{iv.-build-quotes}{%
\subsubsection{\texorpdfstring{iv. Build
\texttt{quote*}s}{iv. Build quote*s}}\label{iv.-build-quotes}}

A comment from Dan Oehm's original code: ``A bit clunky but the path of
least resistance.''

\begin{Shaded}
\begin{Highlighting}[numbers=left,,]
\NormalTok{quote1 }\OtherTok{\textless{}{-}} \FunctionTok{ggplot}\NormalTok{() }\SpecialCharTok{+}
  \FunctionTok{annotate}\NormalTok{(}\AttributeTok{geom =}\StringTok{"text"}\NormalTok{, }
           \AttributeTok{x =} \DecValTok{0}\NormalTok{, }
           \AttributeTok{y =} \DecValTok{1}\NormalTok{, }
           \AttributeTok{label =} \FunctionTok{str\_wrap}\NormalTok{(}\AttributeTok{string =}\NormalTok{ quotes[}\DecValTok{1}\NormalTok{], }\AttributeTok{width =} \DecValTok{40}\NormalTok{),}
           \AttributeTok{family =} \StringTok{"bar"}\NormalTok{, }
           \AttributeTok{fontface =} \StringTok{"italic"}\NormalTok{, }
           \AttributeTok{color =}\NormalTok{ accent, }
           \AttributeTok{size =} \DecValTok{16}\NormalTok{, }
           \AttributeTok{hjust =} \DecValTok{0}\NormalTok{, }
           \AttributeTok{lineheight =} \FloatTok{0.4}\NormalTok{) }\SpecialCharTok{+}
  \FunctionTok{xlim}\NormalTok{(}\DecValTok{0}\NormalTok{, }\DecValTok{1}\NormalTok{) }\SpecialCharTok{+}
  \FunctionTok{ylim}\NormalTok{(}\DecValTok{0}\NormalTok{, }\DecValTok{1}\NormalTok{) }\SpecialCharTok{+}
  \FunctionTok{theme\_void}\NormalTok{() }\SpecialCharTok{+}
  \FunctionTok{coord\_cartesian}\NormalTok{(}\AttributeTok{clip =} \StringTok{"off"}\NormalTok{)}

\NormalTok{quote2 }\OtherTok{\textless{}{-}} \FunctionTok{ggplot}\NormalTok{() }\SpecialCharTok{+}
  \FunctionTok{annotate}\NormalTok{(}\AttributeTok{geom =} \StringTok{"text"}\NormalTok{, }
           \AttributeTok{x =} \DecValTok{0}\NormalTok{, }
           \AttributeTok{y =} \DecValTok{1}\NormalTok{, }
           \AttributeTok{label =} \FunctionTok{str\_wrap}\NormalTok{(}\AttributeTok{string =}\NormalTok{ quotes[}\DecValTok{2}\NormalTok{], }\AttributeTok{width =} \DecValTok{25}\NormalTok{),}
           \AttributeTok{family =} \StringTok{"bar"}\NormalTok{, }
           \AttributeTok{fontface =} \StringTok{"italic"}\NormalTok{,}
           \AttributeTok{color =}\NormalTok{ accent, }
           \AttributeTok{size =} \DecValTok{16}\NormalTok{, }
           \AttributeTok{hjust =} \DecValTok{0}\NormalTok{,  }
           \AttributeTok{lineheight =} \FloatTok{0.4}\NormalTok{) }\SpecialCharTok{+}
  \FunctionTok{xlim}\NormalTok{(}\DecValTok{0}\NormalTok{, }\DecValTok{1}\NormalTok{) }\SpecialCharTok{+}
  \FunctionTok{ylim}\NormalTok{(}\DecValTok{0}\NormalTok{, }\DecValTok{1}\NormalTok{) }\SpecialCharTok{+}
  \FunctionTok{theme\_void}\NormalTok{() }\SpecialCharTok{+}
  \FunctionTok{coord\_cartesian}\NormalTok{(}\AttributeTok{clip =} \StringTok{"off"}\NormalTok{)}

\NormalTok{quote3 }\OtherTok{\textless{}{-}} \FunctionTok{ggplot}\NormalTok{() }\SpecialCharTok{+}
  \FunctionTok{annotate}\NormalTok{(}\AttributeTok{geom =} \StringTok{"text"}\NormalTok{, }
           \AttributeTok{x =} \DecValTok{0}\NormalTok{, }
           \AttributeTok{y =} \DecValTok{1}\NormalTok{, }
           \AttributeTok{label =} \FunctionTok{str\_wrap}\NormalTok{(}\AttributeTok{string =}\NormalTok{ quotes[}\DecValTok{3}\NormalTok{], }\AttributeTok{width =} \DecValTok{25}\NormalTok{),}
           \AttributeTok{family =} \StringTok{"bar"}\NormalTok{, }
           \AttributeTok{fontface =} \StringTok{"italic"}\NormalTok{,}
           \AttributeTok{color =}\NormalTok{ accent, }
           \AttributeTok{size =} \DecValTok{16}\NormalTok{, }
           \AttributeTok{hjust =} \DecValTok{0}\NormalTok{,  }
           \AttributeTok{lineheight =} \FloatTok{0.4}\NormalTok{) }\SpecialCharTok{+}
  \FunctionTok{xlim}\NormalTok{(}\DecValTok{0}\NormalTok{, }\DecValTok{1}\NormalTok{) }\SpecialCharTok{+}
  \FunctionTok{ylim}\NormalTok{(}\DecValTok{0}\NormalTok{, }\DecValTok{1}\NormalTok{) }\SpecialCharTok{+}
  \FunctionTok{theme\_void}\NormalTok{() }\SpecialCharTok{+}
  \FunctionTok{coord\_cartesian}\NormalTok{(}\AttributeTok{clip =} \StringTok{"off"}\NormalTok{)}
\end{Highlighting}
\end{Shaded}

\begin{enumerate}
\def\labelenumi{\arabic{enumi}.}
\setcounter{enumi}{15}
\item
  \textbf{Why do you think the author chose to convert these text
  elements (and also in \texttt{plot\_ufo}, below!) into ggplot objects
  (you may consider returning to this question after you've worked your
  way through all of the code)?}

  \begin{itemize}
  \tightlist
  \item
    The author likely converted these text elements into separate ggplot
    objects so they could be treated as regular plots and integrated
    more feasibly into the final infographic as visual components of the
    same \texttt{geom}* type. By making each quote its own ggplot, each
    \texttt{quote} can be arranged together using other packages like
    \texttt{patchwork} onto the same panel. This allows for easier
    control over styling aesthetics, such as font, color, size, and
    alignment of each individual \texttt{quote}; there is a greater
    control over layout and formatting with the converted text elements.
  \end{itemize}
\end{enumerate}

\hypertarget{v.-build-plot_ufo}{%
\subsubsection{\texorpdfstring{v. Build
\texttt{plot\_ufo}}{v. Build plot\_ufo}}\label{v.-build-plot_ufo}}

\textbf{Note:} Grob stands for \textbf{gr}aphical \textbf{ob}ject. Each
visual element rendered in a a ggplot (e.g.~lines, points, axes, entire
panels, even images) is represented as a grob. Grobs can be manipulated
individually to fully customize plots.

\begin{Shaded}
\begin{Highlighting}[numbers=left,,]
\NormalTok{plot\_ufo }\OtherTok{\textless{}{-}} \FunctionTok{ggplot}\NormalTok{() }\SpecialCharTok{+}
  \FunctionTok{annotation\_custom}\NormalTok{(grid}\SpecialCharTok{::}\FunctionTok{rasterGrob}\NormalTok{(ufo\_image)) }\SpecialCharTok{+}
  \FunctionTok{theme\_void}\NormalTok{() }\SpecialCharTok{+}
  \FunctionTok{theme}\NormalTok{(}
    \AttributeTok{plot.background =} \FunctionTok{element\_rect}\NormalTok{(}\AttributeTok{fill =}\NormalTok{ bg, }\AttributeTok{color =}\NormalTok{ bg) }
\NormalTok{  )}
\end{Highlighting}
\end{Shaded}

\hypertarget{vi.-build-plot_base}{%
\subsubsection{\texorpdfstring{vi. Build
\texttt{plot\_base}}{vi. Build plot\_base}}\label{vi.-build-plot_base}}

\begin{Shaded}
\begin{Highlighting}[numbers=left,,]
\NormalTok{plot\_base }\OtherTok{\textless{}{-}} \FunctionTok{ggplot}\NormalTok{() }\SpecialCharTok{+}
  \FunctionTok{labs}\NormalTok{(}
    \AttributeTok{title =} \StringTok{"UFO Sightings"}\NormalTok{,}
    \AttributeTok{subtitle =} \StringTok{"Summary of over 88k reported sightings across the US"}\NormalTok{,}
    \AttributeTok{caption =}\NormalTok{ caption}
\NormalTok{    ) }\SpecialCharTok{+}
  \FunctionTok{theme\_void}\NormalTok{() }\SpecialCharTok{+}
  \FunctionTok{theme}\NormalTok{(}
    \AttributeTok{text =} \FunctionTok{element\_text}\NormalTok{(}\AttributeTok{family =} \StringTok{"orb"}\NormalTok{, }
                        \AttributeTok{size =} \DecValTok{48}\NormalTok{, }
                        \AttributeTok{lineheight =} \FloatTok{0.3}\NormalTok{, }
                        \AttributeTok{color =}\NormalTok{ accent),}
    \AttributeTok{plot.background =} \FunctionTok{element\_rect}\NormalTok{(}\AttributeTok{fill =}\NormalTok{ bg, }
                                   \AttributeTok{color =}\NormalTok{ bg),}
    \AttributeTok{plot.title =} \FunctionTok{element\_text}\NormalTok{(}\AttributeTok{size =} \DecValTok{128}\NormalTok{, }
                              \AttributeTok{face =} \StringTok{"bold"}\NormalTok{, }
                              \AttributeTok{hjust =} \FloatTok{0.5}\NormalTok{, }
                              \AttributeTok{margin =} \FunctionTok{margin}\NormalTok{(}\AttributeTok{b =} \DecValTok{10}\NormalTok{)),}
    \AttributeTok{plot.subtitle =} \FunctionTok{element\_text}\NormalTok{(}\AttributeTok{family =} \StringTok{"bar"}\NormalTok{, }
                                 \AttributeTok{hjust =} \FloatTok{0.5}\NormalTok{, }
                                 \AttributeTok{margin =} \FunctionTok{margin}\NormalTok{(}\AttributeTok{b =} \DecValTok{20}\NormalTok{)),}
    \AttributeTok{plot.caption =}\NormalTok{ ggtext}\SpecialCharTok{::}\FunctionTok{element\_markdown}\NormalTok{(}\AttributeTok{family =} \StringTok{"bar"}\NormalTok{,}
                                            \AttributeTok{face =} \StringTok{"italic"}\NormalTok{,}
                                            \AttributeTok{color =}\NormalTok{ colorspace}\SpecialCharTok{::}\FunctionTok{darken}\NormalTok{(accent, }\FloatTok{0.25}\NormalTok{),}
                                            \AttributeTok{hjust =} \FloatTok{0.5}\NormalTok{,}
                                            \AttributeTok{margin =} \FunctionTok{margin}\NormalTok{(}\AttributeTok{t =} \DecValTok{20}\NormalTok{)),}
    \AttributeTok{plot.margin =} \FunctionTok{margin}\NormalTok{(}\AttributeTok{b =} \DecValTok{20}\NormalTok{, }\AttributeTok{t =} \DecValTok{50}\NormalTok{, }\AttributeTok{r =} \DecValTok{50}\NormalTok{, }\AttributeTok{l =} \DecValTok{50}\NormalTok{)}
\NormalTok{  )}
\end{Highlighting}
\end{Shaded}

\begin{enumerate}
\def\labelenumi{\arabic{enumi}.}
\setcounter{enumi}{16}
\item
  \textbf{Why does the author render \texttt{plot.caption} using
  \texttt{ggtext::element\_markdown()}, rather than
  \texttt{element\_text()} (like he does for rendering
  \texttt{plot.title} and \texttt{text})?}

  \begin{itemize}
  \tightlist
  \item
    The author uses \texttt{ggtext::element\_markdown()} for the
    \texttt{plot.caption} because the caption contains formatted text or
    specific text elements such as font icons and bold/italic text that
    may not be easily interpreted by \texttt{element\_text()}. As such,
    using \texttt{ggtext::element\_markdown()} is essential for
    maintaining the correct format of the text elements.
  \end{itemize}
\end{enumerate}

\hypertarget{v.-assemble-save}{%
\subsection{V. Assemble \& save}\label{v.-assemble-save}}

\begin{Shaded}
\begin{Highlighting}[numbers=left,,]
\NormalTok{plot\_final }\OtherTok{\textless{}{-}}\NormalTok{ plot\_base }\SpecialCharTok{+}
  \FunctionTok{inset\_element}\NormalTok{(plot\_shape, }\AttributeTok{left =} \DecValTok{0}\NormalTok{, }\AttributeTok{right =} \DecValTok{1}\NormalTok{, }\AttributeTok{top =} \DecValTok{1}\NormalTok{, }\AttributeTok{bottom =} \FloatTok{0.66}\NormalTok{) }\SpecialCharTok{+}
  \FunctionTok{inset\_element}\NormalTok{(plot\_us, }\AttributeTok{left =} \FloatTok{0.42}\NormalTok{, }\AttributeTok{right =} \DecValTok{1}\NormalTok{, }\AttributeTok{top =} \FloatTok{0.74}\NormalTok{, }\AttributeTok{bottom =} \FloatTok{0.33}\NormalTok{) }\SpecialCharTok{+}
  \FunctionTok{inset\_element}\NormalTok{(plot\_day, }\AttributeTok{left =} \DecValTok{0}\NormalTok{, }\AttributeTok{right =} \FloatTok{0.66}\NormalTok{, }\AttributeTok{top =} \FloatTok{0.4}\NormalTok{, }\AttributeTok{bottom =} \DecValTok{0}\NormalTok{) }\SpecialCharTok{+}
  \FunctionTok{inset\_element}\NormalTok{(quote1, }\AttributeTok{left =} \FloatTok{0.5}\NormalTok{, }\AttributeTok{right =} \DecValTok{1}\NormalTok{, }\AttributeTok{top =} \FloatTok{0.8}\NormalTok{, }\AttributeTok{bottom =} \FloatTok{0.72}\NormalTok{) }\SpecialCharTok{+}
  \FunctionTok{inset\_element}\NormalTok{(quote2, }\AttributeTok{left =} \DecValTok{0}\NormalTok{, }\AttributeTok{right =} \DecValTok{1}\NormalTok{, }\AttributeTok{top =} \FloatTok{0.52}\NormalTok{, }\AttributeTok{bottom =} \FloatTok{0.4}\NormalTok{) }\SpecialCharTok{+}
  \FunctionTok{inset\_element}\NormalTok{(quote3, }\AttributeTok{left =} \FloatTok{0.7}\NormalTok{, }\AttributeTok{right =} \DecValTok{1}\NormalTok{, }\AttributeTok{top =} \FloatTok{0.2}\NormalTok{, }\AttributeTok{bottom =} \DecValTok{0}\NormalTok{) }\SpecialCharTok{+}
  \FunctionTok{inset\_element}\NormalTok{(plot\_ufo, }\AttributeTok{left =} \FloatTok{0.25}\NormalTok{, }\AttributeTok{right =} \FloatTok{0.41}\NormalTok{, }\AttributeTok{top =} \FloatTok{0.23}\NormalTok{, }\AttributeTok{bottom =} \FloatTok{0.17}\NormalTok{) }\SpecialCharTok{+} 
  \FunctionTok{plot\_annotation}\NormalTok{(}
    \AttributeTok{theme =} \FunctionTok{theme}\NormalTok{(}
      \AttributeTok{plot.background =} \FunctionTok{element\_rect}\NormalTok{(}\AttributeTok{fill =}\NormalTok{ bg,}
                                     \AttributeTok{color =}\NormalTok{ bg)}
\NormalTok{    )}
\NormalTok{  ) }

\FunctionTok{ggsave}\NormalTok{(}\AttributeTok{plot =}\NormalTok{ plot\_final, }
       \AttributeTok{filename =}\NormalTok{ here}\SpecialCharTok{::}\FunctionTok{here}\NormalTok{(}\StringTok{"outputs"}\NormalTok{, }\StringTok{"ufo\_sightings\_infographic.png"}\NormalTok{), }
       \AttributeTok{height =} \DecValTok{16}\NormalTok{, }
       \AttributeTok{width =} \DecValTok{10}\NormalTok{)}
\end{Highlighting}
\end{Shaded}

\begin{enumerate}
\def\labelenumi{\arabic{enumi}.}
\setcounter{enumi}{17}
\item
  \textbf{Explain how \texttt{plot\_final} is assembled. What do you
  think is the most challenging aspect of arranging all components into
  a single plot?}

  \begin{itemize}
  \item
    Here, plot\_final is assembled by establishing the base plot
    (\texttt{plot\_base}) that provides the main title, subtitle, and
    caption, along with overall styling such as background color, fonts,
    and text size. Individual plot elements like \texttt{plot\_shape},
    \texttt{plot\_us}, \texttt{plot\_day}, and \texttt{plot\_ufo}, as
    well as the text elements (\texttt{quote1}, \texttt{quote2},
    \texttt{quote3}) are then added on as ``layers'' using
    \texttt{inset\_element()}. Each ``layer'' specifies its position
    within the overall plot using coordinates (arguments \texttt{left},
    \texttt{right}, \texttt{top}, and \texttt{bottom}) relative to the
    base plot. Finally, \texttt{plot\_annotation()} is applied to set
    the overall background color referencing \texttt{bg}.
  \item
    Perhaps the most challenging aspect of arranging all components is
    managing the placement and sizing of multiple ``layered'' plots such
    that they do not overlap, are legible, and are balanced visually in
    the overall layout. Because the coordinates are manually set,
    testing individual positions is required to ensure that text,
    charts, and visualizations fit together appropriately and appear
    aesthetically pleasing.
  \end{itemize}
\item
  \textbf{Can you think of one reason the author may have chosen to
  separate the construction of \texttt{plot\_base} and
  \texttt{plot\_final}?}

  \begin{itemize}
  \tightlist
  \item
    It was likely easier to separate \texttt{plot\_base} and
    \texttt{plot\_final} to keep the code organized and flexible,
    defining the overall styling and text once in \texttt{plot\_base}
    and then focusing solely on arranging the individual plots and text
    elements in \texttt{plot\_final}. By creating \texttt{plot\_base}
    first, the author could establish the overall title, caption, and
    styling in a single reusable object. Then, \texttt{plot\_final}
    could be used to focus solely on arranging\textbf{/}layering each
    individual plot without having to redefine the base style each time.
    As a result, this approach was more feasible and reproducible for
    styling and organizing individual elements of the infographic.
  \end{itemize}
\end{enumerate}

\hypertarget{answer-some-final-reflective-questions}{%
\subsection{Answer some final reflective
questions}\label{answer-some-final-reflective-questions}}

\begin{enumerate}
\def\labelenumi{\arabic{enumi}.}
\setcounter{enumi}{19}
\item
  \textbf{During week 2, we discuss
  \href{https://eds-240-data-viz.github.io/course-materials/lecture-slides/lecture2.1-choosing-graphic-forms-slides.html\#/title-slide}{Choosing
  the right graphic form}. Refer to this lecture when answering the
  sub-questions, below:}

  \begin{enumerate}
  \def\labelenumii{\alph{enumii}.}
  \item
    \textbf{What ``perceptual tasks'' (from Cleveland \& McGill's
    heirarchy) must the viewer perform to extract information from these
    visualizations?}

    \begin{itemize}
    \tightlist
    \item
      A viewer might perform some ``perceptual tasks'' to extract
      information from these visualizations. For example, the viewer
      could compare the position along a common, such as reading
      differences in time (hours to days) or interpreting the placement
      of elements within panels (University of Missouri-Saint Louis,
      n.d.). The user could also compare length and area (University of
      Missouri-Saint Louis, n.d.), such as judging the bar lengths or
      focusing on the ``upper'' edge of each bar to assess the total
      sightings per UFO shape. Another key task is comparing the color
      intensity or opacity, which requires judging differences in the
      transparency to infer higher or lower rates of total sightings
      over population over time.
    \end{itemize}
  \item
    \textbf{What task(s) do you think the author wanted to enable or
    message(s) he wanted to convey with these visualizations (see
    lecture 2.1, slide 16 for examples)? Be sure to note at least one
    task / message for each of the three data viz.}

    \begin{itemize}
    \tightlist
    \item
      I believe the author aimed to \uline{visualize spatial patterns}
      in UFO sightings across states over time, using opacity in
      \texttt{plot\_us} to represent differences in sightings per state.
      For \texttt{plot\_shape}, I think the author wanted to
      \uline{allow comparisons of the total number of sightings} for
      each UFO shape, highlighting which shape types were most numerous
      relative to the others. In \texttt{plot\_day}, I believe the
      author wanted to \uline{display temporal trends and highlight
      broader patterns} in UFO sightings across different hours of the
      day and days of the week.
    \end{itemize}
  \item
    \textbf{Name at least one caveat to the ``hierarchy of perceptual
    tasks'' that the author employed to achieve a goal(s) you noted in
    question b?}

    \begin{itemize}
    \tightlist
    \item
      Once caveat to the hierarchy of perceptual tasks is the author's
      use of opacity (color intensity) to encode values. Encoding data
      with color is lower in the hierarchy and may be less precise than
      position or length. While opacity may be effective for showcasing
      overall patterns and highlighting areas of higher or lower
      activity, exact comparisons between values can be difficult. As
      such, viewers can find it difficult to correctly understand small
      differences in opacity, especially with smaller differences in
      total sightings between states in \texttt{plot\_us}, for example.
    \end{itemize}
  \end{enumerate}
\item
  \textbf{Describe two elements of this piece that you find
  visually-pleasing / easy to understand / intuitive. Why?}

  \begin{itemize}
  \tightlist
  \item
    One visually appealing aspect of this piece is the use titles to
    create a U.S. map and represent total UFO sightings by state, which
    makes the spatial distribution intuitive and helps readers easily
    connect sighting frequency with geographic location. This approach
    allows viewers to understand where sightings are most common across
    the country. Another element is the circular \texttt{plot\_day},
    which resembles a clock and helps viewers intuitively visualize when
    UFO sightings are most frequent throughout the day and across the
    week.
  \end{itemize}
\item
  \textbf{Describe two elements of this piece that you feel could be
  better presented in a different way. Why?}

  \begin{itemize}
  \tightlist
  \item
    One element that could be presented differently is the clock-style
    visualization showing when UFO sightings occur. It can be confusing
    to see 12 AM labeled twice, and it would be clearer if 12 PM were
    labeled at the bottom instead. Additionally, the weekday labels are
    difficult to interpret because they are stacked closely together,
    which reduces readability. The opacity scale could also be improved,
    as the least frequently reported UFO shapes are too dark, making it
    harder to distinguish differences among the lower values.
  \end{itemize}
\item
  \textbf{Describe two new things that you learned by interpreting /
  annotating this code. These could be packages, functions, or even code
  organizational approaches that you hadn't previously known about or
  considered.}

  \begin{itemize}
  \tightlist
  \item
    One new thing I learned was how text and annotations can be treated
    as individual ggplot objects and then combined with other plots
    using functions like \texttt{inset\_element()}. This approach to
    building more complex visualizations was unfamilar to me and showed
    how separating text and charts into individual components can make
    layout design more flexible and easier to manage. Another thing I
    learned was how opacity can be scaled for color intensity using
    functions like \texttt{scales::rescale()}. Seeing opacity used
    consistently across different plots to communicate the intensity of
    total sightings helped me better understand how aesthetic mapping
    can reinforce patterns without relying only on position or length.
  \end{itemize}
\item
  \textbf{How, if at all, did you use AI tools to help you interpret
  this code? Describe your approach to using these tools for this
  assignment. In what ways was consulting the documentation more (or
  less) helpful than using AI?}

  \begin{itemize}
  \tightlist
  \item
    While AI was useful in understanding examples of applications of
    unfamiliar functions such as \texttt{inset\_element()}, I found it
    easier to consult the documentation to better understand the
    definition of each function and its specific parameters. For
    example, I found myself consulting the documentation for
    \texttt{font\_add\_google()} to understand the difference between
    the \texttt{name} and \texttt{family} argument in its truest
    definition, while I asked AI to provide examples of different
    situations where \texttt{family} and \texttt{name} were not similar.
  \end{itemize}
\end{enumerate}



\end{document}
